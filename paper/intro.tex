\section{Introduction}
\label{sec:intro}


% \begin{itemize}
% \item Buildings are relying more and more on software 
% \item Software relies on sensors; uses metadata to contextualize the data
% \item Metadata is inputted by human, error-prone; changes over time, becomes stale.
% \item This is perhaps not very common in a single building, but huge problem at scale.
% \item automatic verification is necessary
% \item three types: categorical, spatial, functional
% \item we show how much of each we can get emprirically on the first two; prior work handles functional.
% \end{itemize}

% Buildings represent some of the largest deployments of sensors on physical infrastructure.  Traditionally used
% for historical analysis and supervisory control, these sensors are being used more and more for software-based, model-driven
% control and sophisticated real-time analytics.  The notion of smart buildings consists of a dense deployment of
% sensors in systems and spaces of the building combined with sophisticated software that uses the data collected
% from the deployment to intelligently control and aggregate the information to provide greater insight into the 
% dynamics and pathologies of buildings.  Since buildings consume a large fraction of the energy produced
% in many countries around the world~\cite{}, the move towards smart buildings is accelarating~\cite{}.

% There has been lots of recent work in this area~\cite{BOSS, FIAP, smap, ucsd, bacnetip}

Buildings have become a prime target for cyber-physical systems research, as they consume 40\% of
the energy in the U.S.~\cite{EIA}, are poorly understood,
and offer a rich sensing infrastructure.  Thousands of sensors are embedded throughout 
the building and produce periodic physical measurements. In order to interpret the information, metadata describing 
the placement of sensors is recorded.
However, deployments and their metadata are configured manually.  As such, they are prone to human error.
Moreover, over time sensors are replaced and the physical configuration of the building changes -- walls removed,
new offices set up -- but the metadata describing the new locations are not.  This leads to analytical errors in processes 
that rely on the metadata when interpretting sensor feeds.  For example, model-predictive control processes rely
on the sensors in a specific room or floor~\cite{MPC}.  Because of the size and distributed nature of the deployment, 
it is cumbersome, error-prone, and impractical to maintain accurate metadata about sensor placement over time.  An
automatic processes is needed.

Sensor metadata contains information about the sensor and its context.  It usually units of measure, 
the room it is in, and the system it helps feed.  For example, consider the point name `SODA1R465B_ART'.
\emphSODA1 refers to air-handling unit 1 on soda hall, \emph{R465B} says the sensor is in room 465B, and \emph{ART}
denotes arear-room temperature.  Although not explicitly a unit of measure, the associated data is recorded in degrees
fahrenheit.

Sensor placement information is typically embedded in the name or associated metadata for each sensor in the building.
These are used to group sensors by location.  For example, in our building data, all sensors that contain the string
 `410' in their name are in room 410.  Processes typically group streams in this fashion: using regular-expression matching 
or field-matching queries on the characters in the sensor name or metadata.  If these are not updated to reflect changes
then such group-by query results will not accurately represent true spatial relationships.  
Fontugne et al.~\cite{IOT} observe that spatial associations can be derived empirically.  We start with this approach in our 
work and explore, more deeply, the extent to which it can be used 
as a verification tool for corroborating the groups constructed from character-matching queries.  We refer
to this process as \emph{spatial verification}.

Prior work~\cite{IOT} makes use of a technique called Empirical Mode Decomposition (EMD)~\cite{EMD} to statistically cluster correlated
usage patterns.  Sensors close to each other show strong statistical correlations while sensors further apart show weaker correlations.  
The main parameter in their approach, the correlation threshold, is explored to demonstrate how it relates to characteristic spatial patterns
 in the sensor feeds.  However, they do not characterize the threshold as it relates to physical configuration.
Fontugne et al.~\cite{SBS} expand the work by applying EMD to uncover functional device patterns.  They develop
an unsupervised learning method to model normal usage patterns and apply an anomaly detection algorithm to alert when patterns
have deviated from the norm.  The methodology used in their work divides raw signals into four separate frequency bands
and shows the medium band to carry the most spatial information.

In this paper, we explore the threshold parameter in~\cite{IOT} more deeply, in order to move towards automatic spatial clustering, 
to be used as a form of verification. We use EMD and the intrinsic mode function (IMF) re-aggregation methodology described in~\cite{SBS}, with some modifications, to statistically analyze the threshold parameter
and its relationship to spatial separation in a building.  We explore the hypothesis that \emph{a statistical boundary, analogous to a physical one,
exists and is empirically discoverable}.
We conduct an empirical analysis on the data collected from 15 sensors in 5 rooms over a one-month period.  Our study makes the following contributions:

\begin{itemize}
\item We corroborate the results in~\cite{IOT}, verifying the spatial correlation pattern in a very different building.
\item We characterize the correlation coefficient (corrcoeff) distribution of sensors in the same room and different rooms and validate our existence hypothesis for this preliminary sample.
\item We demonstrate that the statistical boundary between sensors in various rooms converges to a similar value and this value generalizes across rooms in this study.
\item We show the tradeoff between the true and false positive rate inherent to threshold selection. We also show that our method improves the classification accuracy from 80\% to 93.3\%.
\end{itemize}

Our results are promising yet preliminary.  We are able to find a statistical separation across a small number of rooms, quite well.
Our study, however, does not explore the extent to which the physical separation affects the results.  Certainly for rooms that
are far apart we observe a statistical distinction using our methodology.  However, we also find that in some cases, our approach
does not work as well.  We discuss the approach and results in the rest of the paper, followed by a short discussion and future work.


