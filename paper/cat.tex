% cat.tex

\subsection{Problem description}
Among the kind of missing information about a trace can be its categorical association with the S.I. units of measure
and the context in which that sensor was placed.  Again, using the building as a case study, we see that part of the 
naming convention incorporated into the stream tag is the type of sensor and its placement.  Area Room Temperature (ART)
in a stream tag implies both the use of Farenheit to measure temperature and that it is placed inside a zone area,
typically occupied by people.  Similar sensors are placed in different locations in the building, such as outside of it
on the roof, inside vents, and inside pipes, measuring the temperature of the water running through it.  In fact, our
one-year trace contains X temperature sensors in 19 different contexts.

There is a challenge in separating these based on raw readings.  The ranges overlap significantly.

These show significant similarity in the raw trace for temperature traces regardless of context.
\begin{itemize}
\item range overlap
\item KL distance
\end{itemize}

Therefore, the goal is to pick out features that allow us to differentiate between them.  What aspect of the distribution is unique?
Can we pull out certain frequency bands that are particularly informative and representative with respect to the context of
the sensor?

