% Buildings represents one of the richest sensor-enriched environments in the world and
% have become a prime target for energy efficiency.  Work in this area is moving much of
% the monitoring and control tasks to software, to make use of the sensor data.  However,
% the metadata -- used to determine the context of stream -- is manually specified and often
% changes as the building itself evolves.  This distributed information gathering and update
% process is manual and error-prone.  As such, it has become necessary to verify the metadata
% is an automatic fashion.  This paper examines several empirical techniques for verifying various
% kinds of contextual relationships, namely spatial and categorical.  We show how mode decompisition
% can be used to find spatial relationships among sensor streams under certain conditions; doing a
% significantly better job that working with the raw data feeds.  In addition, we show how categorical
% separation can be achieved through supervised learning and offer techniques for dealing with 
% large fractions of missing data -- a fundamental issue sensor-data analysis.


Large buildings contain a rich sensing fabric used for monitoring and control.
With the increased interest in energy efficiency, buildings are becoming more reliant on software processes
to automate control and analytics of the associated sensor data.  These processes rely on the sensor data's
metadata in order obtain the necessary feeds for analysis.  However, the metadata is manually entered and 
does not change with the evolution of the building.  Furthermore, manual maintenance does not scale as the number
of sensors increase with the size of the building.  %An automated verification process is needed.
This paper examines several empirical techniques for verifying various
kinds of contextual relationships, namely spatial and categorical.  We show how mode decomposition
can be used to uncover spatial relationships among sensor streams, where the raw data give no information.
In addition, we show how categorical and placement  
separation can be achieved through supervised learning and offer techniques for dealing with 
large fractions of missing data -- a fundamental issue in sensor-data analysis.